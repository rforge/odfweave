\documentclass[12pt]{article}
\usepackage{amsmath}
\usepackage[pdftex]{graphicx}
\usepackage{color}
\usepackage{xspace}
\usepackage{fancyvrb}
\usepackage{fancyhdr}
\usepackage{lastpage}
\usepackage{algorithm2e}
\usepackage[pdftex,
         colorlinks=true,
         linkcolor=blue,
         citecolor=blue,
         urlcolor=blue]
         {hyperref}

%%%%%%%%%%%%%%%%%%%%%%%%%%%%%%%%%%%%%%%%%%%%%%%%%%%%%%%%%%%%%%%%%%

% define new colors for use
\definecolor{darkgreen}{rgb}{0,0.6,0}
\definecolor{darkred}{rgb}{0.6,0.0,0}
\definecolor{lightbrown}{rgb}{1,0.9,0.8}
\definecolor{brown}{rgb}{0.6,0.3,0.3}
\definecolor{darkblue}{rgb}{0,0,0.8}
\definecolor{darkmagenta}{rgb}{0.5,0,0.5}

%%%%%%%%%%%%%%%%%%%%%%%%%%%%%%%%%%%%%%%%%%%%%%%%%%%%%%%%%%%%%%%%%%

\newcommand{\odfWeave}{{\tt odfWeave}\xspace}
\newcommand{\odfWeaveFunc}{{\tt odfWeave}\xspace}

\newcommand{\bld}[1]{\mbox{\boldmath $#1$}}
\newcommand{\shell}[1]{\mbox{$#1$}}
\renewcommand{\vec}[1]{\mbox{\bf {#1}}}

\newcommand{\ReallySmallSpacing}{\renewcommand{\baselinestretch}{.6}\Large\normalsize}
\newcommand{\SmallSpacing}{\renewcommand{\baselinestretch}{1.1}\Large\normalsize}

\newcommand{\halfs}{\frac{1}{2}}

\setlength{\oddsidemargin}{-.25 truein}
\setlength{\evensidemargin}{0truein}
\setlength{\topmargin}{-0.2truein}
\setlength{\textwidth}{7 truein}
\setlength{\textheight}{8.5 truein}
\setlength{\parindent}{0.20truein}
\setlength{\parskip}{0.10truein}

%%%%%%%%%%%%%%%%%%%%%%%%%%%%%%%%%%%%%%%%%%%%%%%%%%%%%%%%%%%%%%%%%%
\pagestyle{fancy}
\lhead{}
\chead{OdfWeave}
\rhead{}
\lfoot{}
\cfoot{}
\rfoot{\thepage\ of \pageref{LastPage}}
\renewcommand{\headrulewidth}{1pt}
\renewcommand{\footrulewidth}{1pt}
%%%%%%%%%%%%%%%%%%%%%%%%%%%%%%%%%%%%%%%%%%%%%%%%%%%%%%%%%%%%%%%%%%

\title{The OdfWeave Package}
\author{Max Kuhn \\ max.kuhn@pfizer.com}


\begin{document}

\maketitle

\thispagestyle{empty}
	
\section{Introduction}

\odfWeave  is a R package that can be used to format and insert some R objects (e.g. text, data, plots etc) into a file that uses the OpenDocument format (ODF). Effectively, it allows the user to create a self--contained and editable documents that contain R output. \odfWeave  has been mostly developed for text documents created using OpenOffice Writer. Limited testing has been done with other formats (e.g. presentations). 

The package uses the R \texttt{Sweave} function by Friedrich Leisch\footnote{see F. Leisch,  
\href{http://www.ci.tuwien.ac.at/~leisch/Sweave/Sweave-Rnews-2002-3.pdf} {``Sweave, part I: Mixing R and LATEX." R News, 2(3):28-31, December 2002b} for more details about \texttt{Sweave}}, which can currently be used to embed R output into Tex and HTML documents. Much, but not all, of the functionality in Sweave is available in \odfWeave. The \odfWeaveFunc function is a wrapper around the Sweave function, with some file handling and processing functionality. This is due to the substantial differences between the structure of ODF document files and other markup languages, like HTML or Tex (see Section \ref{S:odf} below for details). 

\subsection{Requirements}

\odfWeave  requires files to be in the OpenDocument format, version 1.0 or above. These can be generated by OpenOffice version 2.0 or above. For OpenOffice, \odfWeave  requires that the option ``Size optimization for XML format'' option is {\bf not checked} in the \texttt{Tools:Options:Load/Save:General} menu (i.e. no optimization). The package also requires a utility to zip and unzip compressed files, such as \texttt{unzip}\footnote{a free utility available for many operating systems at  \href{http://www.info-zip.org/} {http://www.info-zip.org/}}, \texttt{Winzip} or \texttt{jar}.

Also, by default, \odfWeaveFunc tries to save images in png format. In Unix and Linux, a png device may not be available. There are three options if this is the case: enable the png device, using the \texttt{bitmap} device (which requires GhostScript) or specify an alternate image format.

To use \odfWeave, the user must have a basic understanding of Sweave.
	
\subsection{Features and Limitations}	
	
Currently, \odfWeave allows users to
\begin{itemize}	
\item Insert R output in--line using the \verb+\Sexpr+ tag
\item Create code blocks using the \verb+<<>>=+ tags to:
   \begin{itemize}	
   \item Insert tables from data frames, matrices and vectors.
   \item Insert bulleted lists from vectors
   \item Embed images from plots using  \verb+<<fig = TRUE>>=+ tags
   \item Embed images created using external programs (i.e. not \texttt{R})
   \end{itemize}
\end{itemize}
The main \texttt{Sweave} features that are not implemented in \odfWeave 
\begin{itemize}	
\item Using \verb+<<>>=+ tags to insert R code and it's unformatted output into the document
\item \texttt{Rtangle} to extracts R code chunks (although the code chunks are invisibly stored in the processed ODF document)
\end{itemize}	
These features, along with expanding table formatting options, are next on the list for development.

	
\section{Calling odfWeave}	

The basic process for creating a document is as follows: create an ODF file (perhaps using OpenOffice) that contains Sweave tags, open R, load the \odfWeave  package and run the \odfWeaveFunc function. The results is another ODF file that has R output instead of the tags.

The \odfWeaveFunc function has several arguments:
\begin{itemize}
   \item {\bf \tt file}: the name of the ODF template that contains the Sweave tags
   \item {\bf \tt dest}: the name of the processed ODF file
   \item {\bf \tt workDir}: the path where the ODF file will be unpacked and processed. If no path is specified, R creates a temporary directory using \verb+tempdir()+
   \item {\bf \tt control}: a list of options for styles, figure sizes and other specifications. The function \verb+odfWeaveControl+ can be used to generate these lists.
\end{itemize}
An example of using \odfWeaveFunc is:
\begin{Verbatim}[fontsize=\footnotesize, frame=single, framesep=5mm, rulecolor=\color{blue}, samepage = true]
library(odfWeave)

# load a built-in list containing document formatting specifications
data(defaultStyles)

# for illustration, change the bullet character
newStyles <- defaultStyles
newStyles$bulletList$bulletChar <- "*"

odfControl <- odfWeaveControl(style = newStyles)
 
odfWeave(
	file    = "/home/user/docTemplate.odt", 
	dest    = "/home/user/finalDoc.odt", 
	control = odfControl)
\end{Verbatim}
Unlike Tex or HTML, if \texttt{docTemplate.odt} contained Sweave chunks to create graphics, the image files are directly embedded in the final document (and the image files are recoverable), so calls to odfWeave produce one file as output by default.

To get more details (and examples) of using \odfWeaveFunc, see the file \texttt{basics.odt} in the \texttt{examples} directory of the package. Example code for processing this file can be found in the help page for \odfWeaveFunc (type \verb+?odfWeave+ at the \texttt{R} prompt).
	
\section{About the OpenDocument Format}\label{S:odf}
	
OpenDocument is a document format that encompasses text documents, spreadsheets, presentations and other types of files. The document extension depends on the document type: odt for text documents, odp for presentations and so on.

There are some good things about the format:
\begin{itemize}
   \item There is a formal specification for the format (see below).
   \item ODF is maintained by a non-profit consortium, the Organization for the Advancement of Structured Information Standards (aka OASIS). No commercial entity owns the format.
   \item The ODF standard is ISO approved.
   \item Documents in the ODF format can be easily converted to a variety of other formats. For example, text documents can be converted to HTML, pdf, Word, plain text, rich text format (RTF) and other formats using OpenOffice. 
   \item An ODF file is a compressed archive of multiple files and directories. All the content for a document is incorporated in a single file.
\end{itemize}	

Some resources for this format are:
\begin{itemize}
\item \href{http://www.oasis-open.org/committees/download.php/12572/OpenDocument-v1.0-os.pdf} {the format specification}\footnote{http://www.oasis-open.org/committees/download.php/12572/OpenDocument-v1.0-os.pdf}
\item ``\href{http://opendocumentfellowship.org/Articles/IntroductionToTheFormatInternals} {Introduction to the format internals}'' by the OpenDocument Fellowship\footnote{http://opendocumentfellowship.org/Articles/IntroductionToTheFormatInternals}
\item ``\href{http://books.evc-cit.info/} {OASIS OpenDocument Essentials}'' by J. David Eisenberg\footnote{http://books.evc-cit.info/}
\end{itemize}


An OpenDocment file is actually a compressed set of files and directories that can be opened with many available applications, such as \texttt{tar}, \texttt{unzip}, \texttt{winzip} or \texttt{jar}. 

There are sub--directories in the archive:
\begin{itemize}
\item {\bf \tt META-INF/} contains manifest.xml, which enumerates the entries in the compressed archive
\item {\bf \tt Pictures/} contains any image files that are included in the document. 
\item {\bf \tt Thumbnails} has images of the rendered document.
\end{itemize}

Additionally, the files contained in the compressed archive include:
\begin{itemize}
\item {\bf \tt content.xml} contains the content of the document (e.g. text paragraphs, tables, etc.)and some formatting.
\item {\bf \tt meta.xml} contains summary information about the document, such as creation date, number of edits, document statistics (e.g. number of words, etc) and the identification of the application that generated the document.
\item {\bf \tt settings.xml} lists the configuration of the document, such as the zoom when opened or printing options.
\item {\bf \tt styles.xml} has formatting information for all of the elements in the document, such as fonts or table formatting.
\end{itemize}
The XML files can be formatted one of two ways. If the ``Size optimization for XML format'' option is checked in (\texttt{Tools:Options:Load/Save:General}), there are almost no linebreaks. If that option is not checked, the XML file is nicely formatted with linebreaks. \odfWeave  requires that this option not be set (although future versions will be able to deal with this).

There are several characters in these XML files that are represented by codes. The characters $>$, $<$, double--quotes, single-quotes and \@, are represented by \texttt{\&gt;}, \texttt{\&lt;}, \texttt{\&quot;}, \texttt{\&apos;} and \texttt{\&amp;}. For example, suppose a document had this code chunk in a text document:
\begin{Verbatim}[fontsize=\footnotesize, frame=single, framesep=5mm, rulecolor=\color{blue}, samepage = true, commandchars=\\\{\}]
\textcolor{red}{<<}figureTest2,fig = TRUE,echo=FALSE,results=hide\textcolor{red}{>>}=
library(lattice)
out \textcolor{red}{<}- densityplot(~randomData, adjust = 1.5)
print(out)
@
\end{Verbatim}
The resulting markup in \texttt{content.xml} would look like:
\begin{Verbatim}[fontsize=\footnotesize, frame=single, framesep=5mm, rulecolor=\color{blue}, samepage = true, commandchars=\\\{\}]
<text:p>\textcolor{red}{&lt;&lt;}figureTest2,fig = TRUE,echo=FALSE,results=hide\textcolor{red}{&gt;&gt;}=</text:p>
<text:p>library(lattice)</text:p>
<text:p>out \textcolor{red}{&lt;}- densityplot(~randomData, adjust = 1.5)</text:p>
<text:p>print(out)</text:p>
<text:p>@</text:p>
\end{Verbatim}

Some other things to think about as you create ODF templates:
\begin{itemize}
\item don't break the $<<>>=$ specifications over multiple lines. 
\item Keep \verb+\Sexpr+ tags one one line (even if they are long)
\item Formatting R code requires some care since a block of R code should have uniform formatting. For example:
   \begin{itemize}
      \item  Typing text in some programs may generate or convert characters to values that cannot be parsed in R. For example, typing double quotes on each side of a word (\verb+"+a word\verb+"+) may be automatically converted to no--standard quotes (e.g. ``a word''). We've tried to make \odfWeaveFunc robust to those changes. 
      \item Mixed formatting of R code is a bad idea. For example,  \texttt{round({\it sqrt(2)}, 2)}, ends up looking like
      
      {\center \small \verb+<text:p>round(<text:span text:style-name="T10">sqrt(2)</text:span>, 2)</text:p>+}
      
      and can be difficult to process and parse.
   \end{itemize}
   \item If a \verb+\Sexpr+ expressions has uniform formatting, the results of that expression should retain the formatting (e.g. {\it \texttt{round(sqrt(2), 2)}} results in {\it \texttt{1.41}}).

\end{itemize}
	
	

%\section{The nuts and Bolts of \odfWeaveFunc}
%
%The following 
%
% \begin{algorithm}[H]
%	\SetLine
%	\restylealgo{plain}
%	\dontprintsemicolon
%	Copy target file to a working directory and unpack\;
%	\For{Each XML file}{
%      Read file into a character vector\; 
%      Create and insert style information into styles.xml\; 
%      \If{Sweave tags}{
%         Remove XML tags around R code in $<<>>=$ chunks\;
%         Write processed XML file to Rnw file\;
%         Sweave Rnw file and output to XML file\;
%         [optinal] Write image files to Pictures directory\;
%      }
%	    Recompress working directory\;
%	}	
%\end{algorithm}
%
%Reading and writing as binary, long lines, UTF-8 encoding
%	
%and example xml file (meta with some tags)
	
\section{Converting ODF to Other Formats}

Using OpenOffice, ODF files can be manually converted to other formats using the  "Save As" or "Export" items in the File menu. To convert documents using a command line interface, there are at last two options:
\begin{itemize}
\item on platforms with the Bash shell, Nathan Coulter has written a script that uses the Python internals that are installed with OpenOffice, called ooconvert.
\item there is a Java class, called  JOOConverter, that is available at \href{http://jooreports.sourceforge.net/} {http://jooreports.sourceforge.net/} that can also convert documents.
\end{itemize}.


\end{document}